\documentclass[11pt,asfal]{amdresume}

\newcommand{\Name}{EL YOUSEFI AHMED}
\newcommand{\job}{Ingénieur Actuaire}
\newcommand{\birthdayDate}{01 oct 2001}
\newcommand{\location}{Rabat, Maroc}
\newcommand{\phoneNumber}{+212 778278749}
\newcommand{\email}{elyousefi.ahmed@outlook.com}
\newcommand{\linkedinName}{Ahmed EL YOUSEFI}
\newcommand{\linkedinLink}{https://www.linkedin.com/in/ahmed-el-yousefi/}
\newcommand{\githubName}{AMD-hub}
\newcommand{\photoLink}{AMD.png}



\begin{document}
\begin{minipage}[c]{0.20\textwidth}
\begin{tikzpicture}
    \node at (0,0) [draw=colorr, fill=colorr, line width=2pt, circle, minimum width=\linewidth, path picture={
        \node at (path picture bounding box.center) {\includegraphics[width=\linewidth]{\photoLink}};
    }] {};
\end{tikzpicture}
\end{minipage}
\vspace{10pt}
\hfill
\hspace{0.1\textwidth}
\begin{minipage}[r]{0.66\textwidth}
    \begin{itemize}[itemsep=0.2cm, topsep=0pt, partopsep=0pt, parsep=0pt]
        \item[] \textbf{\Huge{ \textbf{\uppercase{\Name}}}}
        \item[] \textbf{\Huge{ \textbf{\uppercase{\job}}}}
    \end{itemize}        
\hspace{1cm}
\end{minipage}

\columnratio{0.34}
\setlength{\columnsep}{0.5cm}

\begin{paracol}{2}

\switchcolumn
\section{À PROPOS DE MOI}
Ingénieur Quantitatif et Actuaire spécialisé en modélisation statistique, tarification actuarielle, gestion des risques et analyse assurance, avec expérience en Python, R et SAS.
\vspace{0.1cm}

\section{EXPÉRIENCE}
\subsection{AtlantaSanad Assurance \hfill Casablanca, Maroc}
\subsubsection{Actuaire Non-Vie \dotfill Jan 2025 – Aujourd'hui}
\begin{itemize}
\item Développement du système d'information actuariel.
\item Élaboration du reporting actuariel et tarification des assurances auto.
\end{itemize}

\subsection{La Marocaine Vie \hfill Casablanca, Maroc}
\subsubsection{Stagiaire Actuaire Vie \dotfill Fév 2024 – Juil 2024}
\begin{itemize}
\item Mise en œuvre des normes IFRS 17 sur les contrats de réassurance des produits d'assurance vie (couverture décès emprunteur).
\item Conception du compte de résultat et analyse P\&L pour identifier les leviers clés.
\end{itemize}

\subsection{Financial Risk Solution \hfill Casablanca, Maroc}
\subsubsection{Stagiaire Gestion des Risques \dotfill Juil 2023 – Août 2023}
\begin{itemize}
\item Développement d’un valorisateur d’obligations, valorisation d’un portefeuille et estimation de la Value-at-Risk.
\item Calibration du modèle HJM pour les taux instantanés à terme en utilisant \href{https://www.federalreserve.gov/econres/feds/the-us-treasury-yield-curve-1961-to-the-present.htm}{données U.S. Treasury} (2000–2015).
\end{itemize}

\section{PROJETS}

\subsubsection{Tarification Assurance Auto \dotfill Mar 2023}
\begin{itemize}
    \item Tarification des contrats RC auto via l’approche \textbf{Fréquence-Sévérité} en \textbf{SAS}.
    \item Application d’un \textbf{plafond sur les sinistres importants} pour limiter les pertes extrêmes.
\end{itemize}

\subsubsection{Bibliothèque de Calcul de Provisions \dotfill Mar 2024}
\begin{itemize}
    \item Développement d’une bibliothèque pour le calcul des provisions selon la méthode \textbf{Chain Ladder} en \textbf{Python}.
\end{itemize}

\subsubsection{Optimisation Moyenne-Variance et Application R Shiny \dotfill Jan 2023}
\begin{itemize}
    \item Maximisation du \textbf{ratio moyenne-variance} pour l’optimisation de portefeuille avec \textbf{R}.
    \item Développement d’une application interactive \textbf{R Shiny} pour visualiser la performance du portefeuille et tester différents scénarios.
\end{itemize}

\section{FORMATION}

\subsection{Institut National de Statistique et d’Économie Appliquée \hfill Rabat, Maroc}
\subsubsection{Ingénieur Actuaire et Finance Quantitative \dotfill Oct 2021 – Juil 2024}
\begin{itemize}
\item \textbf{Cours pertinents :} Économétrie, Calcul stochastique, Méthodes quantitatives, Théorie du risque, Méthodes statistiques en finance, Produits à revenu fixe et marchés financiers, Modélisation linéaire et non-linéaire, Analyse de séries temporelles.
\end{itemize}

\subsection{Classes Préparatoires aux Grandes Écoles \hfill Salé, Maroc}
\subsubsection{Mathématiques et Physique \dotfill Sep 2019 – Juil 2021}
\begin{itemize}
\item \textbf{Cours pertinents :} Mathématiques et programmation Python.
\end{itemize}

\newpage \switchcolumn

\section{CONTACT}
\hspace{-0.8cm}
\begin{tabular}{rl}
    \textcolor{colorr}{\faPhone} & \textbf{\phoneNumber} \\
    \textcolor{colorr}{\faGithub} & \textbf{\githubName} \\
    \textcolor{colorr}{\faMapMarker} & \textbf{\location} \\
    \textcolor{colorr}{\faAt} & \href{mailto:\email}{\textbf{\email}} \\
    \textcolor{colorr}{\faLinkedin} & \href{\linkedinLink}{\textbf{\linkedinName}} \\
    \textcolor{colorr}{\faBirthdayCake} & \textbf{\birthdayDate} 
\end{tabular}

\vspace{0.1cm}
\section{Langues}
\hspace{-0.8cm}
\begin{tabular}{lcl}
  \textbf{Arabe}     &  : & \textit{Langue maternelle} \\
  \textbf{Français}  &  : & \textit{Avancé} \\
  \textbf{Anglais}   &  : & \textit{Intermédiaire} \\
\end{tabular}

\vspace{0.1cm}
\section{Hard Skills}
\begin{itemize}
    \item \textbf{Sciences Actuarielles :} Assurance non-vie, Assurance vie, Retraits, Protection, Épargne, Réassurance, Comités exécutifs, Gestion Actif-Passif,\\ 
                                    Solvabilité, IFRS 17
\\ \divider 
    \item \textbf{Ingénierie Financière :} Gestion des \\ risques, Tarification de dérivés, Produits à revenu fixe et modélisation de la courbe des taux, 
                                        Optimisation de portefeuille.
\\ \divider
    \item \textbf{Analyse Statistique :} Régression, Prévision de séries temporelles (ARIMA,\\ SARIMA, etc.), Tests statistiques.
\\ \divider
    \item \textbf{Langages de Programmation :} Python, R, VBA Excel, SAS, C++, SQL
\end{itemize}

\section{Soft Skills}
Esprit analytique, Résolution de problèmes, Communication, Travail en équipe

\section{Loisirs}
\begin{itemize}[left=0.6cm]
    \item[\king] \textbf{Échecs :} \\ Développement de la stratégie et de la réflexion critique.
    \item[\faBook] \textbf{Lecture :} \\ Enrichissement personnel et curiosité intellectuelle.
    \item[\faTv] \textbf{Cinéma :} \\ Appréciation du cinéma comme art émotionnel et de réflexion.
\end{itemize}
 
\end{paracol}

\end{document}
